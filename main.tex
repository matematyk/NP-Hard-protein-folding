\documentclass[leqno,10pt]{article}
\usepackage{algorithm}
\usepackage[noend]{algpseudocode}
\usepackage{hyperref}
\def\Z{\mathbb Z}
\def\Q{\mathbb{Q}}
\def\A{\mathbb{A}}

\makeatletter
\renewcommand{\ALG@name}{Algorytm}
\makeatother

\usepackage{amsmath}
\usepackage{float}
\usepackage{delta}
\def\marg#1{\marginpar{\scriptsize\raggedright#1}}

\begin{document}

\wtyt{Trudność w modelu zawijania białek.}
Modelowanie wielu problemów biologicznych za pomocą matematyki, czy informatyki to bardzo ciekawy dział. Przedewszystkim jest on bardzo. Znane są już w


\waut{Marcin Wierzbiński*}

\marg{Student, Wydział Matematyki, Informatyki i Mechaniki, Uniwersytet Warszawski i SANO}


\vfill
\mtyt{HP model}

\marg{ 
}


\maketitle

\section{Introduction}

\end{document}
